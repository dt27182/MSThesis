\section{Conclusion and Future Work}
In this thesis, I presented a system for digital circuit frontend specification named Automatic Functional Datapath Optimization. In this system, the designer specifies a base functionally correct datapath without any optimizations applied in a RTL like manner and then selects optimization techniques for automatic tools to apply to the base functional datapath. This system of specification is a middle ground approach between RTL specification and HLS specification and tries to capture the conflicting pros of both approaches, namely high designer productivity and the ability to generate highly efficient designs in terms of performance, power, and area at the same time. I implemented two examples of such a system, AutoMArch and AutoFAME, and presented example applications of both.

I hope that AutoMArch can be extended to support more general speculation and that a more general form multi-threading, where combinational logic is replicated as well as the state elements can be explored. I also hope that work can be done to catalogue all of the commonly used digital circuit optimization techniques so that they may one day be implemented automatically
