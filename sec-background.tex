\section{Background}
\subsection{RTL and its Limitations}
\label{section:RTLCons}
In a RTL specification, digital logic designers specify a synchronous digital circuit in terms of data flow between synchronous state elements such as flip-flops and SRAM memory blocks. There are generally two types of variables in RTL languages: registers and wires. Registers correspond to synchronous state elements and wires correspond to the output of intermediate combinational logic blocks. The designer specifies the state elements needed by declaring registers and specifies the combinational logic that connects the state elements by using conditional assignment operators along with arithmetic operators on registers and wires declared within the design. This RTL specification is then synthesized into a gate level specification through a synthesis tool and then fed into the physical design portion of the IC design flow. 

The first limitation of RTL specification is that mainstream RTL languages, such as Verilog and VHDL, are based on discrete event driven simulation languages and the syntax and semantics of discrete event driven simulation languates are not entirely natural for specifying digital logic. For example, state elements are not explicitly defined, but are instead inferred from variables that are specified to update when a clock signal transitions. This hurts source code readability because in order to understand a design specified by the RTL, the user needs to not only parse the semantics of the RTL, but also mentally reason about the circuit constructs implied by the RTL. The semantics of discrete event driven simulation languages also allows the designer to create un-synthesizable contructs in RTL. This leads to RTL designs that pass tests in simulation, but cannot be physically realized as a circuit.

The second limitation of RTL specification is that although RTL specification frees the designer from specifying every transistor or logic gate and their connections, RTL specification is still very tedious and verbose as the designer has to worry about exactly how every state elements updates every clock cycle. This also makes it easy for the designer to lose the big picture algorithmic view of the design as the designer is bogged down in the implementation details. The functional behavior of the design is obfuscated by the implementation details of the optimizations applied to the design.

\subsection{Chisel}
The first limitation of RTL mentioned in the above section can be addressed through use of RTL languages that are not based on dicrete event driven simulation languages. One example of this is Chisel\cite{Bachrach:2012}, a RTL language developed by UC Berkeley. Unlike Verilog and VHDL, it specifies digital circuits through explicit circuit component construction, not inference based on a discrete event simulator semantics. This makes it more intuitive to specifiy digital design through RTL because there is a very simple mapping from the RTL constructs to physical circuit constructs. Additionally, all RTL designs written Chisel that pass tests in simulation can be physically realized as a circuit.

While Chisel does not address the second limitation of RTL specification, it will be used to specify the base functional datapath for the digital hardware frontend system proposed in this thesis.

\subsection{HLS and its Limitations}
High Level Synthesis (HLS) has been an attempt to address both limitations of RTL specification mentioned in \ref{section:RTLCons} by going to a higher level of specification than RTL. In this paradigm, the designer specifies the logic design as a data flow graph through sequential variable updates in a C like language, which frees the designer from having to specify the cycle by cycle operation of a specific datapath. Then a HLS tool maps the dataflow graph to a datapath based on performance, power, and area constraints and synthesizes a gate-level description of the design from the high level specification, which is then fed into the physical design portion of the IC design flow. 

Unfortunately, designs synthesized from HLS specifications generally fail to have acceptable performance, power, and area characteristics compared to equivalent designs synthesized from RTL specifications. Much effort has been put into making the synthesis process produce more optimal designs, but rapid breakthroughs are unlikely given that several subproblems in the process have been proven to be NP-Hard\cite{McFarlan:1990}.
