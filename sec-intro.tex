\section{Introduction}
Over the past 30 years, progress in programming languages has greatly increased the productivity of software design by moving low level resource allocation and optimization tasks away from the programmer to the compiler or interpreter.  In contrast, increase in productivity of digital hardware logic design has largely stalled after the transition from schematic based specification to to Register Transfer Level (RTL) specification languages such as Verilog or VHDL for front end specification.

While RTL specification succeeds in providing the digital logic designer with a specification that is higher level than transistor level or gate level schematic specification, RTL specification is still quite a low abstraction level and requires the designer to provide a very detailed specification of the design. High Level Synthesis (HLS) has been an attempt to increase the productivity of digital logic designers by going to a much higher level of specification than RTL and automatically synthesizing all of the details of the design. However, the higher abstraction level of HLS comes at the cost of unsatisfactory performance, area, and power characteristics due to inefficiencies introduced in the automatic logic synthesis process.

In this thesis, I propose a system in which the designer specifies a basic functional datapath in a RTL like manner and selects optimizations that should be applied to the design, which will then be automatically applied through a tool. This middle ground method increases digital logic designer productivity without producing designs with unacceptable performance, power, and area characteristics like HLS. I implement two examples of such a system, AutoMArch and AutoFAME, and I describe their implementation and example applications.

