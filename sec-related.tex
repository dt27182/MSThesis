\section{Background}
Over the past 30 years, progress in programming languages has greatly increased the productivity of software design by moving low level resource allocation and optimization tasks away from the programmer to the compiler or interpreter.  In contrast, increase in productivity of digital hardware logic design has largely stalled after the transition from schematic based specification to to Register Transfer Level (RTL) specification languages such as Verilog or VHDL. Mention Gap between RTL and HLS

High Level Synthesis (HLS) has been an attempt to increase the productivity of digital logic designers by going to a higher level of specification than RTL. In this paradigm, the designer specifies the logic design as a data flow graph through sequential variable updates in a C like language, which frees the designer from having to specify the cycle by cycle operation of a specific datapath. Then a HLS tool maps the dataflow graph to a datapath based on performance/power/area constraints and synthesizes a gate-level description of the design from the high level specification, which is then fed into the physical design portion of the IC design flow. Unfortunately, HLS has largely failed to gain acceptance by digital logic designers because the designs synthesized from the high level specifications generally fail to have acceptable performance/area/power characteristics compared to equivalent designs synthesized from RTL specifications. Much effort has been put into making the synthesis process produce more optimal designs, but rapid breakthroughs are unlikely given that several subproblems in the process have been proven to be NP-Hard\cite{McFarlan:1990}.

Nurvitadhi et al~\cite{hoe:syn}, \cite{hoe:mult} present separate tools T-spec for transactional datapath specification and T-piper for automatic pipeline synthesis. T-spec is used to describe transactional datapaths as state elements and acyclic next-state logic blocks that updates those state elements. Users manually annotate all the state elements and next-stage logic blocks with the pipeline stage number. T-piper analyzes the T-spec design to identify RAW hazards and generate hazard resolution logic. My tool is different in that the datapath and pipeline specification are both within the same language. Additionally, my tool is capable of automatically assigning datapath components to pipeline stage numbers.
