\section{Related Work}
Older work in the area \cite{Kroening:2001-95}\cite{Matthews:1999-106}\cite{Marinescu:2001-105}\cite{Higgins:2005-76}\cite{Mishra:2004-111} generally focus on automatically verifying pipelined designs based on a given ISA or describe automatic pipelining system that still require manual intervention on part of the pipelining process.

Nurvitadhi et al~\cite{hoe:syn} present separate tools T-spec for transactional datapath specification and T-piper for automatic pipeline synthesis. T-spec is used to describe transactional datapaths as state elements and acyclic next-state logic blocks that updates those state elements. Users manually annotate all the state elements and next-stage logic blocks with the pipeline stage number. T-piper analyses the T-spec design to identify RAW hazards and generate hazard resolution logic. \cite{hoe:mult} extends the above tool to be able to generate multi-threaded in-order pipelines. AutoMArch is different in that the pipeline specification does not require a entirely separate language. Additionally, my tool is capable of automatically assigning datapath components to pipeline stage numbers, which saves a lot of designer specification and potentially produces more balanced pipelines.

Galceran-Oms \cite{AutoPipeElastic} present a method to automatically pipeline synchronous elastic systems. The paper presents a set of provably correct transformations on synchronous elastic systems and show that it is possible to pipeline a micro architecture by applying a sequence of such transformations. AutoMArch is different than the system presented here because it applies pipelining and multi-threading optimizations to ordinary finite state machines with a few IO restrictions rather than to synchronous elastic systems.
